\section{Visualización de Software}

La visualización de software refiere al uso de diversos elementos (texto, 
gráficos 2D y 3D, diagramas, imágenes, vídeos, entre otros) para representar algún
aspecto de un programa y facilitar su comprensión.\cite{PetreDeQuincey06,Chen06,GomezHenriquez01}
Estos aspectos u objetos de interés pueden ser abstracciones de componentes, de
un sistema completo e incluso del comportamiento en tiempo en ejecución de los
mismos. \cite{TeysereCampo09}

Las visualizaciones pueden ayudan al programador con tres procesos cognitivos
\cite{ButlerAlmond93}:
durante un proceso de descubrimiento o exploratorio, en donde el usuario no
sabe específicamente qué está buscando;
durante un proceso de toma de decisión o analítico, en el cuál el usuario sabe lo que
está buscando y sólo necesito encontrarlo;
durante un proceso descriptivo, en el cuál es conocido el patrón que aparece en los
datos pero necesita una visualización acorde para expresarlo.

Estos procesos cognitivos que se pueden ver beneficiados por las visualizaciones, suelen
emplearse en diferentes tareas que puede llevar a cabo un programador.
Dentro de esta lista de tareas \cite{MalleticMarcusCollard02} encontramos: desarollo,
debugging, pruebas, mantenimiento y detección de errores, re-ingeniería, ingeniería reversa,
administración del proceso de software, y marketing.
En las primeras etapas del desarrollo de un nuevo sistema o aplicación, las visualizaciones
permiten la colaboración entre desarrolladores y ayudar en la definición y validación de los
requerimientos.
A medida que se avanza en la construcción del software, las herramientas permiten validar
los avances y encontrar defectos.
En el caso de mantenimiento, ayudan en la comprensión del programa, entender su diseño y
así plantear su evolución \cite{PetreDeQuincey06}.

Algoritmos, programas y sistemas son los diferentes alcances que se consideran
dento de la disciplina de Visualización de Software \cite{PriceBaeckerSmall93,Myers90}.
La Visualización de Sistemas se emplea para representar el sistema en su conjunto,
a través de los diferentes módulos que lo componen.
En la visualización de Programas, se busca mostrar el comportamiento de los mismos
y la relación entre sus componentes \cite{BenAri01}.
Por último, la Visualización de Algoritmos se utiliza principalmente en el campo de la enseñanza,
para demostrar el funcionamiento de algoritmos y estructuras de datos.
Así mismo, tal como denota la expresión 
\begin{center}
    $Alg. Visualization \subseteq Prog. Visualization \subseteq System Visualization$
\end{center}
las técnicas implementadas en un determinado alcance, pueden ser implementadas
también en un conjunto de mayor nivel de abstracción \cite{BeronHenriquesPereiraUzal07}.

La taxonomía de Price, Baecker y Small \cite{PriceBaeckerSmall93}, establece que
existen seis categorías de atributos para los sistemas de visualización.
Dentro de las mismas podemos encontrar:
\begin{enumerate}
    \item \textbf{Alcance:} refiere al espectro de programas que pueden ser analizados.
    \item \textbf{Contenido:} corresponde al conjunto de propiedades del sistema que se pueden
    extraer y visualizar.
    \item \textbf{Forma:} relacionada a la salida de los resultados de la visualización.
    \item \textbf{Método:} determina cómo está especificada la visualización.
    \item \textbf{Interacción:} referida la manera en la que el usuario pueda controlar e
    interactuar con la visualización.
    \item \textbf{Efectividad:} grado de adecuación con el cual se comunica la información
    al usuario.
\end{enumerate}
Del listado anterior, la categoría y por lo tanto la característica más importante corresponde
a la \textit{efectividad}.

Dependiendo de la fuente de datos, las visualizaciones pueden estar basadas en
estructuras estáticas (cuando se obtienen de vistas del código fuente), o de
información extraída en tiempo de ejecución (vistas dinámicas, así sean de flujo de datos
o control de flujo). \cite{PetreDeQuincey06}
Las estrategias para la obtención de los datos se discuten a continuación.
