\section{Introducción}

La \textit{Comprensión de Programas} es una disciplina de la Ingeniería de Software
cuyo objetivo es proveer \textit{Modelos, Métodos, Técnicas} y \textit{Herramientas}
para facilitar el estudio y entendimiento de los sistemas de software \cite{BeronHenriquesPereira10}.

La importancia de la Comprensión de Programas radica en que es una de las demandas cognitivas
más importantes, dentro del conjunto necesario para llevar adelante las tareas implicadas
en el proceso de desarrollo y sus distintas fases \cite{PetreDeQuincey06}.
La correcta aplicación de los modelos, métodos, técnicas y herramientas propuestas por
la Comprensión de Programas, hacen que el desarrollador pueda localizar y entender rápidamente
los elementos empleados para soportar una funcionalidad específica, disminuyendo el tiempo requerido
para la realización de una tarea, impactando positivamente en los costos asociados a la
modificación, mantenimiento y evolución de un sistema \cite{BeronHenriquesPereira10}.

La forma que tienen los programadores de entender un sistema, componente o pieza de código fuente
puede ser descrita a través de diferentes modelos, los cuales componen la base para las herramientas
de comprensión de programas \cite{BeronHenriquesPereiraUzal07}.
Éstos son denominados \textit{modelos cognitivos}.
La información necesaria para aplicar los modelos cognitivos, se obtiene a través de
\textit{métodos de extracción de la información}.
Para facilitar la compresión y el proceso establecido en el modelo cognitivo, se utilizan
diferentes técnicas de \textit{visualización de software}, y así procesar mejor la
información extradía por distintos métodos.
Por último, diferentes \textit{estrategias para la inter-conexión de dominios} pueden ser
aplicadas para que el programador asocie el dominio del problema, con el dominio
del programa bajo estudio.

Las siguientes secciones, desarrollan sobre los conceptos nombrados previamente.
