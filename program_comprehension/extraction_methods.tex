\section{Métodos de Extracción de la Información}

Para poder lograr comprender una pieza de código fuente o programa bajo estudio, un
programador debe ser capaz de poder obtener información del mismo.
El tipo de información que el usuario necesita, depende de la tarea que tenga que llevar
adelante respecto al software.
Dentro la disciplina de Comprensión de Programas, para extraer la información requerida 
por los diferentes procesos cognitivos, se utilizan dos tipos de análisis de código:
\textit{Análisis Estático} y \textit{Análisis Dinámico}.

El \textbf{análisis estático} del software consiste en obtener información sobre el 
sistema de interés, pero sin ejecutarlo.
Los \textit{analizadores sintácticos} operan a nivel de código fuente, empleando
técnicas tanto de \textit{pattern matching} como de construcción de árboles de sintáxis
abstracta (\textit{Abstract-Syntax Tree}).
Los \textit{analizadores semánticos} trajaban de una manera diferente, ya que consideran
la secuencia en la que pueden darse los estados de un programa, en una ejecución.
La premisa principal consiste en que analizar los posibles estados de un programa en un
punto determinado, son suficientes para probar propiedades estáticas del mismo.
\cite{Cousot77}

El \textbf{análisis dinámico} corresponde a la extracción de propiedades de un sistema
en ejecución.
Estas propiedades son obtenidas al examinar una o más ejecuciones de un programa, a
través de diferentes técnicas.
Generalmente, se obtiene a través de la instrumentación del código fuente, y así
analizar los aspectos que son de interés; y es muy dependiente del input que reciba
el programa.
\cite{Ball99}
