\section{Problema}

"Software ages as this social and organisational context evolves: teams change,
knowledge decays, documentation falls out of date, intentions and rationale are forgotten over time
(Ball and Eick, 1996)."

"Software development is a complex undertaking, involving a number of related tasks: specification,
design, implementation, testing, debugging, maintenance, modification, re-engineering. Each of those
tasks involves a number of cognitive demands, such as search, comprehension, analysis, problem
solving, representation. It’s no wonder that software development is so often described as complex.
The complexity derives from the nature of the problems addressed, the diversity of the tasks involved,
the nature of the artefacts produced, the changing social and organisational environment in which it is
conducted, and the impact of time on environment, needs, teams, and artefacts."\cite{PetreDeQuincey06}

\section{Conceptos Preliminares}

\section{Solución}

\section{Contribución}
\subsection{Objetivos Generales}
\subsection{Objetivos Específicos}

\section{Organización del informe}
El presente informe se compone de una sucesión de 7 capítulos, organizados para introducir
el problema (Capítulo 1), establecer el marco teórico para darle soporte (Capítulos 2, 3 y 4),
presentar la herramienta que aporta una solución (Capítulo 5), comentar los experimentos y
las conclusiones obtenidas (Capítulos 6 y 7).

Dichos capítulos son descritos a continuación:

\paragraph[]{Capítulo 1: Introducción} Se determina el problema que da origen al informe,
junto con los conceptos premilinares y la solución propuesta.
Así mismo, se definen la contribución del presente trabajo junto con los objetivos generales
y específicos, y se presenta la organización correspondiente al informe.

\paragraph[]{Capítulo 2: Comprensión de Programas} \textit{TBD}

\paragraph[]{Capítulo 3: Métricas de Calidad en Software} Se revisa el estado del arte
en cuanto a las métricas de Calidad de Software, haciendo hincapié en las relacionadas al análisis
estático de código.

\paragraph[]{Capítulo 4: Análisis de Identificadores} Se ahonda sobre los conceptos relacionados
a los indicadores y sus propiedades; y se recorren con detalle los algoritmos para el procesamiento
de los mismos, tanto para división como expansión.
Se explica cada uno de los algoritmos, haciendo énfasis en los implementados en el presente trabajo.

\paragraph[]{Capítulo 5: Herramienta} Se presenta la herramienta, describiendo su estructura y
el proceso completo desde el análisis de código hasta la obtención de las métricas de interés.
También se explican las particularidades del lenguaje soportado por la herramienta.

\paragraph[]{Capítulo 6: Experimentos y Resultados} Se determinan cuáles fueron los experimentos
realizados, así como también los resultados de los análisis de los mismos.

\paragraph[]{Capítulo 7: Conclusiones} Se presentan las conclusiones extraídas 
después de la ejecución de los experimentos y el análisis de los resultados.
Además, se plantean trabajos futuros a partir del actual.
