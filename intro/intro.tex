\section{Problema}

"Software ages as this social and organisational context evolves: teams change,
knowledge decays, documentation falls out of date, intentions and rationale are forgotten over time
(Ball and Eick, 1996)."

"Software development is a complex undertaking, involving a number of related tasks: specification,
design, implementation, testing, debugging, maintenance, modification, re-engineering. Each of those
tasks involves a number of cognitive demands, such as search, comprehension, analysis, problem
solving, representation. It’s no wonder that software development is so often described as complex.
The complexity derives from the nature of the problems addressed, the diversity of the tasks involved,
the nature of the artefacts produced, the changing social and organisational environment in which it is
conducted, and the impact of time on environment, needs, teams, and artefacts."\cite{PetreDeQuincey06}

"Software quality assesment is an important task in software engineering, because it can
reduce maintenance effort, a main cost in software development.
A typical way to asses quality facets, such as program comprehension, is to use software
measures.
Software measures are computed based on properties of source code"\cite{Feigenspan2011}.

"Spinellis (2003) observes that programmers are poor at choosing meaningful identifier names
because they find it difficult to concurrently manage the expression of programming constructs
along with the managing of natural language description, say to invent identifier names.
This activities appears too taxing on the programmer as it requires rapid context shifting
between two different cognitive processes that use the same underlying apparaturs"\cite{Relf04}.

"Identifier names constitute the majority of tokens in source code" \cite{DeiBenbockPizka05}.

"Identifiers are the primary source of conceptual information for program
comprehension" \cite{RajlichWilde02}.

"In \cite{Lawrie2006}, detail an empirical study which found identifier names composed
of dictionary words were more easily understood than those composed of abbreviations
or single letters".

Importancia de trabajar e identificar aquellos fragmentos de código que no siguen un
determinado lineamiento.
Distancia entre el nombre del indicador y el mejor obtenido, como métrica para identificar
módulos que en los que se puede ver afectada la comprensibilidad del software.

"Identifier names are created by designers and programmers and reflect their
understanding, cognition and idiosyncrasies" \cite{HostOstvold07}

"A longitudinal study of identifier names by Lawrie et al. \cite{Lawrie2007} showed that
identifier name quality has improved during the las thirty years. 
The same study also found
that identifiers in propietary source code tipically contained more domain-specific
abbreviations than open source code.
However, the study also found that identifiers change little following its initial period
of software development".

\section{Conceptos Preliminares}

\section{Solución}

\section{Contribución}
\subsection{Objetivos Generales}
\subsection{Objetivos Específicos}

\section{Organización del informe}
El presente informe se compone de una sucesión de 7 capítulos, organizados para introducir
el problema (Capítulo 1), establecer el marco teórico para darle soporte (Capítulos 2, 3 y 4),
presentar la herramienta que aporta una solución (Capítulo 5), comentar los experimentos y
las conclusiones obtenidas (Capítulos 6 y 7).

Dichos capítulos son descritos a continuación:

\paragraph[]{Capítulo 1: Introducción} Se determina el problema que da origen al informe,
junto con los conceptos premilinares y la solución propuesta.
Así mismo, se definen la contribución del presente trabajo junto con los objetivos generales
y específicos, y se presenta la organización correspondiente al informe.

\paragraph[]{Capítulo 2: Comprensión de Programas} \textit{TBD}

\paragraph[]{Capítulo 3: Métricas de Calidad en Software} Se revisa el estado del arte
en cuanto a las métricas de Calidad de Software, haciendo hincapié en las relacionadas al análisis
estático de código.

\paragraph[]{Capítulo 4: Análisis de Identificadores} Se ahonda sobre los conceptos relacionados
a los indicadores y sus propiedades; y se recorren con detalle los algoritmos para el procesamiento
de los mismos, tanto para división como expansión.
Se explica cada uno de los algoritmos, haciendo énfasis en los implementados en el presente trabajo.

\paragraph[]{Capítulo 5: Herramienta} Se presenta la herramienta, describiendo su estructura y
el proceso completo desde el análisis de código hasta la obtención de las métricas de interés.
También se explican las particularidades del lenguaje soportado por la herramienta.

\paragraph[]{Capítulo 6: Experimentos y Resultados} Se determinan cuáles fueron los experimentos
realizados, así como también los resultados de los análisis de los mismos.

\paragraph[]{Capítulo 7: Conclusiones} Se presentan las conclusiones extraídas 
después de la ejecución de los experimentos y el análisis de los resultados.
Además, se plantean trabajos futuros a partir del actual.
