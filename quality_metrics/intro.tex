\section{Introducción}

"Identifier names constitute the majority of tokens in source code" \cite{DeiBenbockPizka05}.

"Identifiers are the primary source of conceptual information for program
comprehension" \cite{RajlichWilde02}.

"Identifier names are created by designers and programmers and reflect their
understanding, cognition and idiosyncrasies" \cite{HostOstvold07}

"The impact of low quality identifier names on program comprehension is reasonably
well understood \cite{DeiBenbockPizka05,Lawrie2007,Lawrie2006}, but little is known
about the extent to which the quality of identifier names might influence the quality
of source code" \cite{ButlerWemelingerYu10}.

\textbf{"a low-cost heuristic to identify potentially problematic regions of source code"}

"Existing research on source code readability focuses on the contribution the
components of source code make to readability" \cite{Buse2008}.

"A longitudinal study of identifier names by Lawrie et al. \cite{Lawrie2007} showed that
identifier name quality has improved during the las thirty years. 
The same study also found
that identifiers in propietary source code tipically contained more domain-specific
abbreviations than open source code.
However, the study also found that identifiers change little following its initial period
of software development".

"In \cite{Lawrie2006}, detail an empirical study which found identifier names composed
of dictionary words were more easily understood than those composed of abbreviations
or single letters".

"Buse and Weimer \cite{Buse2008} developed a readability metric for Java derived from measurements 
of, among others, the number of parentheses and braces, line length, the number
of blank lines, and the number, frequency and length of identifiers.
Using machine learning, the readability metric was trained to agree with the judgement 
of human source code readers.
Buse and Weimer found a significant statistical relationship between the readability of 
methods and the presence of defects found by FindBugs in open source code bases. 
Although their work makes a link between readability and software quality, their notion
of readability ignores the quality of identifier names".

Se ignora el contenido semántico.

"The predictive probability associated with each relationship illustrates the utility
of the identifier flaws as light-weight classifiers for source code quality"
\cite{ButlerWemelingerYu10}.

REVISAR OTROS PAPERS

