\section{Introducción}

"Identifier names constitute the majority of tokens in source code" \cite{DeiBenbockPizka05}.

"Identifiers are the primary source of conceptual information for program
comprehension" \cite{RajlichWilde02}.

"Identifier names are created by designers and programmers and reflect their
understanding, cognition and idiosyncrasies" \cite{HostOstvold07}

"The impact of low quality identifier names on program comprehension is reasonably
well understood \cite{DeiBenbockPizka05,Lawrie2007,Lawrie2006}, but little is known
about the extent to which the quality of identifier names might influence the quality
of source code" \cite{ButlerWemelingerYu10}.

\textbf{"a low-cost heuristic to identify potentially problematic regions of source code"}

"Existing research on source code readability focuses on the contribution the
components of source code make to readability" \cite{Buse2008}.

"A longitudinal study of identifier names by Lawrie et al. \cite{Lawrie2007} showed that
identifier name quality has improved during the las thirty years. 
The same study also found
that identifiers in propietary source code tipically contained more domain-specific
abbreviations than open source code.
However, the study also found that identifiers change little following its initial period
of software development".

"In \cite{Lawrie2006}, detail an empirical study which found identifier names composed
of dictionary words were more easily understood than those composed of abbreviations
or single letters".

"Buse and Weimer \cite{Buse2008} developed a readability metric for Java derived from measurements 
of, among others, the number of parentheses and braces, line length, the number
of blank lines, and the number, frequency and length of identifiers.
Using machine learning, the readability metric was trained to agree with the judgement 
of human source code readers.
Buse and Weimer found a significant statistical relationship between the readability of 
methods and the presence of defects found by FindBugs in open source code bases. 
Although their work makes a link between readability and software quality, their notion
of readability ignores the quality of identifier names".

Se ignora el contenido semántico.

"The predictive probability associated with each relationship illustrates the utility
of the identifier flaws as light-weight classifiers for source code quality"
\cite{ButlerWemelingerYu10}.

"Software quality is not defined in terms of quality attributes but instead must be
inferred from characteristics that correlate to quality attributes and defect attributes.
One of these quality attributes is the readability of the software.
However, the software engineer, who is ultimately responsible for software quality,
is not supported well by their formal education, the software engineering culture, the
existence of useful tools and where these tools do exist by their limited up-take by industry.
Human cognitive limitations similarly frustrate the development of readable source code.
Software characteristics have been identified, which correlate well to source code readability.
One of these software characteristics, which have been supported by empirical research,
is the choice of identifier name".\cite{Relf04}

"ISO-9126 defines a set of six quality attributes: efficiency, functionality, maintanability,
portability, realiability and usability.
However, standards like ISO-9126 only offer top-level guidance that define software quality,
they do not prescribe how to insert quality into software (Dromey, 1995).
Some objective measure of software quality is required in order to categorise acceptable
practices (Vollman, 1993) and software quality will not improve until there is a comprehensive
definition available"\cite{Relf04}.

"Spinellis (2003) observes that programmers are poor at choosing meaningful identifier names
because they find it difficult to concurrently manage the expression of programming constructs
along with the managing of natural language description, say to invent identifier names.
This activities appears too taxing on the programmer as it requires rapid context shifting
between two different cognitive processes that use the same underlying apparaturs"\cite{Relf04}.

"A software characteristic that has the potential to improve software quality is the choice of
identifier name, and this is particularly so in large software systems.
identifier-naming sytle guidelines, supported by empirical evidence and generally accepted by
software professionals to direct towards improved source code readabiity are candidates for
automation by a tool.
Such an automated tool could make visibile aspects of software quality that are less keenly
perceived by the novice programmer and could assist in ther education along the path to
expert status"\cite{Relf04}.

"Software quality assesment is an important task in software engineering, because it can
reduce maintenance effort, a main cost in software development.
A typical way to asses quality facets, such as program comprehension, is to use software
measures.
Software measures are computed based on properties of source code"\cite{Feigenspan2011}.

REVISAR OTROS PAPERS

Importancia de trabajar e identificar aquellos fragmentos de código que no siguen un
determinado lineamiento.
Distancia entre el nombre del indicador y el mejor obtenido, como métrica para identificar
módulos que en los que se puede ver afectada la comprensibilidad del software.
